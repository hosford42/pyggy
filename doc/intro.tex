\chapter{Introduction}

PyGgy is a python package for generating parsers and lexers in python.
The PyGgy distribution contains two tools:

\begin{tableii}{l|l}{exception}{Tool}{Description}
  \lineii{PyLly (Pronounced "pile-ey")}{ A lexer generator that generates
            DFA tables for lexing tokens.}
  \lineii{PyGgy (Pronounced "piggy")}{A parser generator that generates 
            SLR tables for a GLR parsing engine.}
\end{tableii}

The PyLly program is used to pregenerate tables for a finite state
machine from a lexer specification.  There is a lexer engine that uses
the tables to tokenize an input stream.

The PyGgy program is used to pregenerate parser tables from a parser
specification.  There is a GLR parsing engine that uses the tables 
to parse a stream of input tokens.  Because GLR parsing is used, the
parser can deal with arbitrary grammars, even if they are recursive
or ambiguous.

PyGgy is self hosting -- PyGgy parsers and PyLly lexers are used
to process the specification files used by both PyGgy and PyLly.

This is version 0.4 of PyGgy.  This release contains minor bug fixes
to the 0.3 release.
This is an alpha release;  the public interfaces are subject to
change, and there may be bugs in the code.  This
version of PyGgy is placed in the public domain.  This means anyone
can do anything with it with no restrictions whatsoever.

This version has been tested with Python 2.3.3.

