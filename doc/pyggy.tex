\documentclass{manual}

\title{PyGgy Manual}

\input{boilerplate}

\makeindex

\begin{document}

\release{0.4}
\setshortversion{0.4}
\author{Tim Newsham}
\authoraddress{Email: \email{newsham@lava.net}}
\date{October 15, 2004}
\maketitle

\ifhtml
\chapter*{Front Matter\label{front}}
\fi

\begin{abstract}

% XXX macro for PyGgy and PyLly ?

\noindent
PyGgy is a Python package for generating and executing lexers and
parsers.  The package is complete enough to be used in its own
construction.  PyGgy is in the public domain.

\end{abstract}

\tableofcontents

\chapter{Introduction}

PyGgy is a python package for generating parsers and lexers in python.
The PyGgy distribution contains two tools:

\begin{tableii}{l|l}{exception}{Tool}{Description}
  \lineii{PyLly (Pronounced "pile-ey")}{ A lexer generator that generates
            DFA tables for lexing tokens.}
  \lineii{PyGgy (Pronounced "piggy")}{A parser generator that generates 
            SLR tables for a GLR parsing engine.}
\end{tableii}

The PyLly program is used to pregenerate tables for a finite state
machine from a lexer specification.  There is a lexer engine that uses
the tables to tokenize an input stream.

The PyGgy program is used to pregenerate parser tables from a parser
specification.  There is a GLR parsing engine that uses the tables 
to parse a stream of input tokens.  Because GLR parsing is used, the
parser can deal with arbitrary grammars, even if they are recursive
or ambiguous.

PyGgy is self hosting -- PyGgy parsers and PyLly lexers are used
to process the specification files used by both PyGgy and PyLly.

This is version 0.4 of PyGgy.  This release contains minor bug fixes
to the 0.3 release.
This is an alpha release;  the public interfaces are subject to
change, and there may be bugs in the code.  This
version of PyGgy is placed in the public domain.  This means anyone
can do anything with it with no restrictions whatsoever.

This version has been tested with Python 2.3.3.


	% version and maturity
	% references to parsing intro
	% features
	% home page
\chapter{Quick Start}

This chapter introduces the PyGgy package and illustrates its use
with examples.  The information in this chapter is intentionally
incomplete.  For detailed information on using PyLly or PyGgy refer
to the API and specification reference chapters.

\section{Installing}
To install the PyGgy package, unpack the archive, change to the
\file{pyggy} directory and run the setup program:

\begin{verbatim}
$ python setup.py install
\end{verbatim}


\section{Using PyLly}
PyLly reads in a specification file for a lexer and generates tables
that can be used by a lexer to tokenize a stream of data.  The first
step in using PyLly is to construct a specification file.  This
file specifies how to pull tokens out of an input stream.  
An example of a simple specfile is given in \file{test1.pyl}:

\begin{verbatim}
INITIAL :
    "abc*" :    return "pattern1"
    "d|efg" :   return "pattern2"
    "^abx" :    return "pattern3"
    "\n" :      return              # ignore newlines
\end{verbatim}

This specification states that if an \code{a} then a \code{b} and then an
arbitrary number of \code{c} characters are encountered, the lexer should
return the string with a token value of \code{"pattern1"}.  
Similarly if a \code{d} or \code{e} followed
by the characters \code{fg} are seen, then a token value of \code{"pattern2"} 
should be returned.
If the characters \code{abx} are encountered at the start of a line then
\code{"pattern3"} is returned.  Finally if the newline character is seen,
no token is returned.  Any other characters cause an error token
to be returned.

Once a specification file has been made, it is a simple matter to
use it.  The \file{examples/test1.py} program provides an example:

\begin{verbatim}
import pyggy

l,tab = pyggy.getlexer("test1.pyl")
l.setinput("-")
while 1 :
    x = l.token()
    if x is None :
        break
    print x, l.value
\end{verbatim}

The \code{getlexer} function takes care of parsing the lexer specification
file, generating tables for a lexer, loading the tables and constructing
a lexer.  It returns the constructed lexer and a handle on the generated
lexer table module.  The example makes use of the lexer by first specifying
an input file for the lexer to read from and then by calling the
\code{token} method to retrieve each successive token.  The \code{setinput}
method is used to specify which file to read input from.  In this case
the special name \file{-} is used which denotes that input should
come from \code{stdin}.  The \code{token} method returns the next token
value from the input stream when called.  It returns \code{None} when
the input source has been exhausted.  An auxiliary value stored in
the \code{value} member contains the value of the token (which is usually
a string of the characters that make up the token).

The example can be run as follows:

\begin{verbatim}
$ echo "abccccdfghabx" | python test1.py
pattern1 abcccc
pattern2 dfg
#ERR# h
pattern1 ab
#ERR# x
\end{verbatim}

Notice that the lexer returns error tokens for the unrecognized
patterns.


\section{Using PyGgy}

PyGgy reads in a specification file for a parser and generates
tables that can be used by a parsing engine to parse a stream of
tokens.  The first step in using PyGgy is to construct a specification
file which specifies the grammar to be parsed.  An example of a simple
spec file is given in \file{examples/test2.pyg} (see \file{examples/test2.pyl}
for the lexer that goes along with it):

\begin{verbatim}
# This grammar is ambiguous
E -> E PLUS E
    | E TIMES E
    | ID
    ;
\end{verbatim}

This specifies a grammar with one non-terminal (\code{E}) with three
productions (\code{E -> E PLUS E}, \code{E -> E TIMES E} and \code{E -> ID}) 
and three terminals (tokens \code{PLUS}, \code{TIMES} and \code{ID}).

Building a parser from a specification file is similar to building
a lexer from a PyLly specification file.  The example in
\file{example/test2.py} illustrates this:

\begin{verbatim}
import pyggy

[...]

# instantiate the lexer and parser
l,ltab = pyggy.getlexer("test2.pyl")
p,ptab = pyggy.getparser("test2.pyg")
l.setinput("-")
p.setlexer(l)

# parse the input
tree = p.parse()
if tree == None :
    print "error!"
else :
    print "parse done: ", exprstr(tree)
    # if you have dot, try uncommenting the following
    #pyggy.glr.dottree(tree)
\end{verbatim}

The \code{getparser} function builds parser tables and a parser
in a similar manner as the \code{getlexer} method previously
discussed.  It returns both the parser and the generated module
containing the parser tables.  Once the parser is specified, its
input source is specified with the \code{setlexer} method.   Finally
the \code{parse} method is called to parse the token stream from
the lexer.

The \code{parse} method parses the tokens from the lexer and returns 
a parse tree.  The tree has a slightly different shape than might be expected
because the parse engine can
parse ambiguous grammars.  The root of the tree is a
\code{pyggy.glr.symbolnode} instance.  This instance refers to one of the
terminals or non-terminals in the grammar.  It has a list
of the possible productions that are derived by that symbol in
the \code{possibilities} member.  If the parse in unambiguous, there
will be exactly one item in the \code{possibilities} list.
Each possibility is a \code{pyggy.glr.rulenode} instance.  The \code{rulenode}
instance represents the left hand side of a production and has
members \code{rule} specifying which rule was matched and \code{elements}
which is a list of all of the parsed items in the right hand side.  These
elements are \code{symbolnode} instances.

To clarify, consider the code from \file{examples/test2.py}:

\begin{verbatim}
def singleexprstr(kids) :
    if len(kids) == 1 :
        return kids[0].sym[1]
    else :
        return "(%s %s %s)" % (exprstr(kids[0]), kids[1].sym[1], exprstr(kids[2]))

def exprstr(e) :
    res = []
    for p in e.possibilities :
        res.append(singleexprstr(p.elements))
    if len(res) == 1 :
        return res[0]
    else :
        return "[" + string.join(res, " or ") + "]"
\end{verbatim}
 
The \code{exprstr} function is called to convert a parsed expression tree
into a string.  The \code{exprstr} function is called with a \code{symnode} that
always references the non-terminal \code{"E"}.
Exprstr converts each of
the possible parses of \code{"E"} into a string by calling \code{singleexprstr}.
For each of the possible \code{rulenodes}, it calls the \code{singleexprstr}
function with a list of the right hand side elements.  The \code{singleexprstr}
function converts this right hand side list into a string.  If there
is only one item in the right hand side, it must be an identifier, and
the identifiers value is retrieved from the symbol information.  Otherwise
there are three children, two expressions and an operator.  The expressions
are converted to strings and joined into a single string with the operator
between them.

Its informative to see what the output of this function looks like:

\begin{verbatim}
$ echo "a+b*c" | python run2.py
parse done:  [((a + b) * c) or (a + (b * c))]
\end{verbatim}

Notice that there were two possible parses of this string.  If you
have GraphViz installed, edit the \file{examples/test2.py} example and 
uncomment 
the line \code{pyggy.glr.dottree(tree)} and rerun the previous test case.
You will be shown a graphical representation of the parse tree.
In the graph, the \code{symnodes} show up in red and the \code{rulenodes} show
up in black.  You can also see the value of the symbol in the \code{symnode}
and the rule in the \code{rulenode}.  Notice that the \code{symnode} for each
non-terminal has a \code{sym} value of the non-terminal name and each terminal
has a \code{sym} value that is a tuple of the token name and the token
value.  Also note that the each \code{rulenode} has a rule that is a tuple
of the name of the left hand side, the number of elements in the right
hand side (which is also the number of items in its \code{elements} variable)
and the index of the production in the grammar.

The graphical view of the parse {\em tree} makes it obvious that its
not a parse {\em tree} at all!  The parsing engine
makes use of shared nodes whenever possible to avoid an exponential
blowup in the number of nodes during an ambiguous parse.  The
parse {\em tree} will truely be a tree if there is a unique parse.  The 
parse {\em tree} may have cycles if there is a production which can 
derive itself without consuming any input.  If there are no such 
productions, there will be no cycles.

Like the lexer specification file, the grammar specification file can
be used to specify actions to be performed.  These actions are not performed
during parsing, as is traditionally done, but can be invoked after the
parse is complete.  The \file{example/pyg\_calc.py} example illustrates
this.  This is a small calculator test case based on the example
from the PLY web site 
(\citetitle[http://systems.cs.uchicago.edu/ply/example.html]{http://systems.cs.uchicago.edu/ply/example.html}).
The \file{example/pyg\_calc.pyg}
file specifies the grammar and actions:

\begin{verbatim}
%left TIMES DIVIDE;
%left PLUS;
%right UNARYMINUS;

statement -> NAME EQUALS expression :
        names[kids[0]] = kids[2]
    | expression :
        print kids[0]
    ;

expression -> expression PLUS expression :
        return kids[0] + kids[2]
    | %prec(PLUS) expression MINUS expression :
        return kids[0] - kids[2]
    | expression TIMES expression :
        return kids[0] * kids[2]
    | expression DIVIDE expression :
        return kids[0] / kids[2]
    | %prec(UNARYMINUS) MINUS expression :
        return -kids[1]
    | LPAREN expression RPAREN :
        return kids[1]
    | NUMBER :
        return kids[0]
    | NAME :
        if not kids[0] in names :
                print "Undefined name '%s'" % kids[0]
            return 0
        return names[kids[0]]
    ;
\end{verbatim}

Each production in the grammar specifies a block of code following
the final colon.  After the input is parsed into a parse tree
the actions can be applied to the parse tree with the \code{proctree}
function as is done in \file{examples/pyg\_cal.py}:

\begin{verbatim}
import sys
import pyggy

# build the lexer and parser
l,ltab = pyggy.getlexer("pyg_calc.pyl")
p,ptab = pyggy.getparser("pyg_calc.pyg")
p.setlexer(l)

while 1:
    sys.stdout.write("calc > ")
    line = sys.stdin.readline()
    if line == "" :
        break

    l.setinputstr(line)
    try :
        tree = p.parse()
    except pyggy.ParseError,e :
        print "parse error at '%s'" % e.str
        continue
    pyggy.proctree(tree, ptab)
\end{verbatim}

\code{proctree} walks the tree in a depth-first manner and
at each node representing a production in the grammar, runs
the action associated with that production.  When running the
code the \code{kids} argument contains a list of the right
hand side values of the production.



\section{Visualization}
Because many of the data structures used and returned by PyGgy and
PyLly are complex, PyGgy provides hooks to visualize them using
the AT\&T GraphViz program \file{dotty}.  If you do not have
the GraphViz program, you will still be able to use the PyGgy package
but you may not be able to access all the debugging information.
GraphViz is available free of charge at \citetitle[http://www.research.att.com/sw/tools/graphviz/]{http://www.research.att.com/sw/tools/graphviz/}.

To use graphviz with PyGgy, make sure that \file{dotty} is in your path,
and invoke PyGgy or PyLly with a high debug value (for example,
specifying \code{debug=3} as an argument to \code{getlexer} or 
\code{getparser}) or use the function \code{pyggy.glr.dottree} to
display trees returned from the \code{parse} method.


\section{Using PyLly with PLY}
Lexers generated by PyLly can be used by PyGgy parsers or by
other parser packages.  The lexer interface was designed to be
similar enough to PLY's lexer interface that PyLly lexers can
be used as-is.  For other parsers, a small wrapper may class
may be needed around the lexer to provide the proper interface.

The \file{example/ply\_calc.py} example illustrates how to use
a PLY parser with a PyLly lexer.  This is a small calculator test 
case based on the example from the PLY web site 
(\citetitle[http://systems.cs.uchicago.edu/ply/example.html]{http://systems.cs.uchicago.edu/ply/example.html}).
The \file{example/ply\_calc.pyl} file specifies the lexer:

\begin{verbatim}
code :
    tokens = (
        'NAME','NUMBER',
        'PLUS','MINUS','TIMES','DIVIDE','EQUALS',
        'LPAREN','RPAREN',
    )

    lineno = 1

    class Tok :
        def __init__(self, l, type) :
            self.type = type
            self.lineno = lineno
            self.value = l.value
        def __str__(self) :
            return "Tok(%s,%r,%d)" % (self.type, self.value, self.lineno)

definitions :
    NAME    "[[:alpha:]_][[:alnum:]_]*"
    NUMBER  "[[:digit:]]+"

INITIAL :
    "\+" :  return Tok(self, "PLUS")
    "-" :   return Tok(self, "MINUS")
    "\*" :  return Tok(self, "TIMES")
    "/" :   return Tok(self, "DIVIDE")
    "=" :   return Tok(self, "EQUALS")
    "\(" :  return Tok(self, "LPAREN")
    "\)" :  return Tok(self, "RPAREN")

    "{NAME}":   return Tok(self, "NAME")
    "{NUMBER}":
        try :
            self.value = int(self.value)
        except ValueError :
            print "Integer value too large", self.value
            self.value = 0
        return Tok(self, "NUMBER")

    "[[:blank:]]" : return
    "\n+" :
        global lineno
        lineno += len(self.value)
        return

    "." :
        print "Illegal character '%s'" % self.value
        return
\end{verbatim}

Each token's value is represented by a \code{Tok} instance
which adheres to PLY's token interface.  The specification
also defines a list of tokens in the \code{tokens} variable
as needed by PLY.  The \file{examples/ply\_calc.py} script
builds the lexer and uses it to feed a PLY parser:

\begin{verbatim}
# build the lexer
import pyggy
l,lexer = pyggy.getlexer("ply_calc.pyl")
tokens = lexer.tokens

[...]

import yacc
yacc.yacc()

while 1:
    sys.stdout.write("calc > ")
    line = sys.stdin.readline()
    if line == "" :
        break

    l.setinputstr(line)
    yacc.parse(lexer=l)
\end{verbatim}


\section{More Examples}
For more examples of the use of PyLly and PyGgy refer to the
code in the \file{examples} directory.  In addition to several
small examples there are the beginnings of a C parser in the
\code{examples/ansic} directory.  This example is based on
the ANSI C test case from the D-Parser found at
\citetitle[http://dparser.sourceforge.net/d/tests/ansic.test.g]{http://dparser.sourceforge.net/d/tests/ansic.test.g}.

The most complete
example of using PyGgy and ultimate reference for the behavior
of PyGgy is found in the source code of PyGgy.  The files
\file{pylly.pyl} and \file{pyggy.pyl} specify the spec file lexers 
while the files \file{pylly.pyg} and \file{pyggy.pyg} specify their
grammars.  The files \file{pylly.py} and \file{pyggy.py} complete
the functionality of the lexer and parser specification reading
code.  They are slightly complicated by their use of a lower level
API which allows them to use pre-generated lexer and parser
tables.  


	% visualizer
	% using pylly with other parsers
\chapter{API Reference}

This section describes the public interfaces provided by
the PyGgy package.

\section{\module{pyggy}}
\declaremodule{extension}{pyggy}
\modulesynopsis{The PyGgy and PyLly parsing and lexing package}
\moduleauthor{Tim Newsham}{newsham@lava.net}

The \module{pyggy} module is the container for the entire PyGgy
and PyLly package.  Importing this module loads in the ``simple''
API.   This API consists of a number of utility functions and
exception classes.

\begin{funcdesc}{pyggy.generate}{fname, targ, debug=0, forcgen=0}
This function takes in filenames specifying a specification file
and an output file name.  If the output file does not exist or
is older than the specification file, it processes the specification
file and generates tables into the target file.  The input specification
must be a \file{.pyl} or \file{.pyg} file.  If the \code{debug} argument
is specified, increased debugging information is emitted while
processing the spec file.  For
a description of the debug levels see the documentation on the
\module{pyggy.pyggy} and \module{pyggy.pylly} modules.  If \code{forcegen}
is true, the specification file is processed whether or not it is
newer than the target file.

If an invalid specification file is specified, \class{pyggy.ApiError}
is raised.  Any exception raised by \method{pyggy.pyggy.parsespec} or
\method{pyggy.pylly.parsespec} may also be raised.
\end{funcdesc}

\begin{funcdesc}{pyggy.getlexer}{specfname, debug=0, forcegen=0}
This function generates a lexer table module from a lexer spec file, imports
the module and returns a \class{pyggy.lexer.lexer} instance and the
module.  The arguments \code{debug} and \code{forcegen} have the
same meaning as in the \function{generate} function.

This function can raise any of the exceptions raised by \function{generate}.
\end{funcdesc}

\begin{funcdesc}{pyggy.getparser}{specfname, debug=0, forcegen=0}
This function generates a parser table module from a parser spec file,
imports the module and returns a \class{pyggy.glr.GLR} instance
and the module.  The arguments \code{debug} and \code{forcegen} have the
same meaning as in the \function{generate} function.

This function can raise any of the exceptions raised by \function{generate}.
\end{funcdesc}


\begin{funcdesc}{pyggy.proctree}{t, gram, allowambig=0}
This function post-processes a parse tree previously returned
from a call to a \method{parse} method.  It takes as arguments
the parse tree and the module containing the parse tables.  The
optional argument \code{allowambig} is used to specify that ambiguous
parses are allowed, otherwise a \class{pyggy.AmbigParseError} is
raised if any ambiguities are encountered in the parse tree.

The function walks the parse tree in a bottom-up fashion executing
the semantic action code for each production used in the derivation.
For each action executed, the list of the values from the right
hand side of the production are passed in.  These values are either
from the \code{value} fields of tokens, or the values previously
returned by other action code functions.  If ambiguous parses are
disallowed, each right hand side element is represented by the
element's unique value.  In the case that ambiguous parses are allowed
and an ambiguity is found in the parse tree, the ambiguous right
hand side element will be represented with a \class{pyggy.glr.symnode}
instance whose \code{possibilities} field is a list of the alternate
values. 

The \function{proctree} function returns the value associated with
the start symbol which is at the root of the parse tree.

This function may raise \class{pyggy.AmbigParseError} if an ambiguous
parse is detected or \class{pyggy.ApiError} if an invalid tree is passed in.
This function alters the parse tree as it operates on it.
\end{funcdesc}

\begin{excdesc}{pyggy.Error}
A superclass of all PyGgy generated exceptions.
\end{excdesc}

\begin{excdesc}{pyggy.InternalError}
An error in the inner workings of PyGgy or PyLly.  This is a type
of \class{pyggy.Error}.
\end{excdesc}

\begin{excdesc}{pyggy.ApiError}
This error specifies that there is an error in the way that
PyGgy or PyLly is being used.  This is a type of \class{pyggy.Error}.
\end{excdesc}

\begin{excdesc}{pyggy.LexError}
This error specifies that there was an error while lexing an
input source.  This exception is not currently used since the lexer
currently returns an error token on errors.  This exception is a
type of \class{pyggy.Error}.
\end{excdesc}

\begin{excclassdesc}{pyggy.ParseError}{str, tok}
This error specifies that an error occured during the parsing of
a token stream.  It stores a description of the error in \code{str}
and the token that caused the error in \code{tok}.  This exception
is a type of \class{pyggy.Error}.
\end{excclassdesc}

\begin{excdesc}{pyggy.AmbigParseError}
This error specifies that an ambiguity was present in a parse tree
when ambiguities are disallowed.  It is a type of \class{pyggy.ApiError}.
\end{excdesc}


\section{\module{pyggy.lexer} -- The lexing engine}
\declaremodule{extension}{pyggy.lexer}
\modulesynopsis{The PyLly lexing engine}
\moduleauthor{Tim Newsham}{newsham@lava.net}

\begin{datadesc}{pyggy.lexer.TOK\_ERR}
\code{TOK\_ERR} is returned by the lexer whenever a character is encountered
which cannot be lexed.
\end{datadesc}

\begin{classdesc}{pyggy.lexer.lexer}{lexspec}
The lexer class provides an extensible lexer class capable of lexing
tokens from an input source based on tables generated by PyLly.
The \code{lexspec} argument is passed in from a generated table module.
It should be a tuple made up of a DFA table, a list of start states,
a list of actions for each accepting state, a list of actions to
be performed at the end of file and a dictionary of character classes.
The format of the \code{lexspec} argument is subject to change.

The \class{lexer} class provides lexing from input strings or from
files.  The class can be subclassed to provide lexers with different
input behaviors.

\begin{memberdesc}{value}
The \code{value} member holds the value associated with the most
recently returned token.
\end{memberdesc}

\begin{methoddesc}{setinputstr}{str}
Sets the input source to be the characters in the \code{str} argument.
\end{methoddesc}

\begin{methoddesc}{setinput}{fname}
Sets the input to source to be the characters from the specified file.
A filename of \file{-} indicates data should come from \code{stdin}.
\end{methoddesc}

\begin{methoddesc}{token}{}
Returns the next token lexed from the input stream.  If there are no
more tokens left, the value \code{None} is returned.  If an error occurs
the value \code{pyggy.lexer.TOK\_EOF} is returned.  In the future an
exception may be raised in this situation.  Before a token is returned,
the \code{value} member is set to the value specified in the action
code for the token, or to a string of the characters in the token.
\end{methoddesc}

\end{classdesc}



\section{\module{pyggy.srgram}}
\declaremodule{extension}{pyggy.srgram}
\modulesynopsis{Shift-Reduce table handler}
\moduleauthor{Tim Newsham}{newsham@lava.net}

\begin{classdesc}{pyggy.srgram.SRGram}{srspec}
This class encapsulates the tables of a shift-reduce parser to
isolate the parsing engine from the details of the table implementation.
It's constructor takes a single argument which should be the
grammar table specification from a generated grammar table module.
The specification is a tuple of a GOTO table, an ACTION table and
a list of semantic actions.
The format of this specification is subject to change.
\end{classdesc}



\section{\module{pyggy.glr } -- The PyGgy parsing engine}
\declaremodule{extension}{pyggy.glr}
\modulesynopsis{The PyGgy parsing engine}
\moduleauthor{Tim Newsham}{newsham@lava.net}

This module implements the Generalized-LR parsing engine and provides
the supporting data structures and helper functions.


Parse trees are composed of alternating levels of \class{symnode}
and \class{rulenode} instances.  The root of the parse tree is
a \class{symnode} for the start symbol of the grammar.

\begin{classdesc}{pyggy.glr.symnode}{sym, possibilities, cover}
The \class{symnode} class is used to represent terminals and non-terminals
in the parse tree while parsing.  It's \code{sym} field is either
a tuple of the token name and value for terminals or a non-terminal
symbol.  The \code{possibilities} field holds a list of all possible
derivations of the current symbol.  Each element in \code{possibilities}
is a \class{rulenode} instance.
\end{classdesc}

\begin{classdesc}{pyggy.glr.rulenode}{rule, elements, cover}
The \class{rulenode} class is used to represent a production used
in the derivation.  It's \code{rule} member specifies which production
in the grammar the \class{rulenode} represents.  It is a tuple
of the left hand side symbol, a count of right hand side elements and
the production number in the grammar.  The \code{elements} member
is a list of \class{symnode}s for the right hand side of the
production.
\end{classdesc}

\begin{funcdesc}{pyggy.glr.dottree}{tree, printcover=0}
This is a helper for visualizing a parse tree.  It uses the
\file{dotty} program to show a graphical representation of the
parse tree \code{tree}.  If \code{printcover} is true, the
token positions each tree node covers is also displayed.
\end{funcdesc}

\begin{classdesc}{pyggy.glr.GLR}{gram}
The \class{GLR} class implements the GLR parsing engine.  It takes
in a single argument which is a \class{pyggy.srgram.SRGram} instance.

\begin{methoddesc}{setlexer}{lex}
Sets input to come from the lexer \code{lex}.  This must be called
before parsing is started.
\end{methoddesc}

\begin{methoddesc}{parse}{}
Parses the stream of tokens from the designated lexer and return
a parse tree.

This method may raise \class{pyggy.PaseError} or \class{pyggy.InternalError}.
\end{methoddesc}
\end{classdesc}



\section{\module{pyggy.pylly} -- Lexer generation}
\declaremodule{extension}{pyggy.pylly}
\modulesynopsis{Lexer generator}
\moduleauthor{Tim Newsham}{newsham@lava.net}

This module implements the lexer generator.  It is responsible for
reading in a lexer spec file (in \file{.pyl} format), generating
finite state machines and emitting tables for the machines into a
Python module.

This module can be accessed from the command line or through a call
from Python.  To run from the command line:

\begin{verbatim}
$ python pylly.py [-d debuglevel] infile.pyl outfile.py
\end{verbatim}

\begin{funcdesc}{pyggy.pylly.parsespec}{fname, outfname, debug=0}
The \function{parsespec} function causes the input file \code{fname}
to be processed and finite state machine tables to be generated to
\code{fname}.  If \code{debug} is set, increasing amounts of diagnostic
output will be emitted.  The debug levels (especially higher-levels)
are subject to change but are currently:

\begin{tableii}{l|l}{exception}{Level}{Description}
  \lineii{0} {Output a count of ambiguities in the lexer.}
  \lineii{1} {Output detailed diagnostics of the generated lexer.}
  \lineii{2} {Show the DFAs constructed for each start state with \file{dotty}.}
  \lineii{3} {Show the NFA constructed from the spec file with \file{dotty}.}
  \lineii{10}{Show the parse tree from the spec file with \file{dotty}.}
\end{tableii}

This function may raise \class{pyggy.SpecError} if there are any
errors in the spec file or \class{pyggy.InternalError}.
\end{funcdesc}



\section{\module{pyggy.pyggy} -- Grammar generation}
\declaremodule{extension}{pyggy.pyggy}
\modulesynopsis{Parser generator}
\moduleauthor{Tim Newsham}{newsham@lava.net}


This module implements the parser generator.  It is responsible for
reading in a parser spec file (in \file{.pyg} format), generating
a shift-reduce tables and emitting the tables into a
Python module.

This module can be accessed from the command line or through a call
from Python.  To run from the command line:

\begin{verbatim}
$ python pyggy.py [-d debuglevel] infile.pyg outfile.py
\end{verbatim}

\begin{funcdesc}{pyggy.pyggy.parsespec}{fname, outfname, debug=0}
The \function{parsespec} function causes the input file \code{fname}
to be processed and shift-reduce tables to be generated to
\code{fname}.  If \code{debug} is set, increasing amounts of diagnostic
output will be emitted.  The debug levels (especially higher-levels)
are subject to change but are currently:

\begin{tableii}{l|l}{exception}{Level}{Description}
  \lineii{0} {Output a count of ambiguities in the parser.}
  \lineii{1} {Output detailed diagnostics of the generated parser.}
  \lineii{2} {Turn on debugging in the parser generator engine and show precedence relations.}
  \lineii{3} {Show the LR0 state machine}
  \lineii{11}{Show the parse tree from the spec file with \file{dotty}.}
  \lineii{12}{Show the cover while showing the parse tree.}
\end{tableii}

This function may raise \class{pyggy.SpecError} if there are any
errors in the spec file or \class{pyggy.InternalError}.
\end{funcdesc}


    % quick api
    % errors
    % pylly api
    % lexer api
    % pyggy api
    % parser api
\chapter{The PyLly spec file}

The PyLly spec file specifies regular expressions to match and code
to execute when a pattern is matched.   The file itself is comprised
of a number of sections. 

\section{The Code section}
The first section type is the \code{code} section.  It specifies code to
be copied directly to the output file (unindented).  The \code{code} section
is started with the word \code{code} followed by a colon and a newline.  All
the indented code that follows is copied directly to the output file.
For example:

\begin{verbatim}
code :
    def printjoined(l) :
        print " ".join(l)

    printjoined["this","is","a","test"])
\end{verbatim}

If more than one code section appears in the spec file, the code from
each section is gathered together and copied out to the output file.

\section{The Definition section}
The second type of secion type is the \code{definitions} section.  It provides
a convenient way to define often-used patterns.  It is started with
the word \code{definitions} followed by a colon and a newline.  Each line
in the definitions file specifies a pattern name and a pattern seperated
by an arbitrary number of spaces.  Patterns are always regular expressions 
enclosed in quotes in PyLly [Regular expressions will be discussed
more fully later].  For example:

\begin{verbatim}
definitions:
    ID      "[A-Za-z_][A-Za-z0-9_]*"
    IDPAIR  "{ID} {ID}"
\end{verbatim}

This example defines the pattern named \code{ID} as a letter or underscore
followed by zero or more letters, numbers or underscores.  This pattern
can be used latter in other regular expressions by enclosing them in
braces as illustrated in the definition of \code{IDPAIR} as two identifiers
seperated by a space.  All named patterns are self contained; there
is never a need to enclose a named pattern within parenthesis to avoid
unusual effects.

There may be multiple definitions sections within the spec file.  Definitions
may be used in any patterns following the definition.

\section{Starting States}
The final type of sections in PyLly are starting state sections.  Each
start state section is indicated with a list of start state names seperated
by commas followed by a colon and a newline.  Each line within the
indented section specifies a pattern and code to execute when the pattern
is matched.  The code is specified either as a single line of code
immediately following a colon, or the indented block of code following
a colon and a newline.  For example:

\begin{verbatim}
INITIAL :
    "{ID}" :
        idcnt += 1
        return "Identifier"
    "abc*" : return "pattern1"
\end{verbatim}

This section specifies two patterns within the \code{INITIAL} start state.
When the first pattern is matched, two lines of code are executed.
When the second line is matched, just one line of code is executed.

Start states specify seperate state machines.  Most lexers will only
need one start state, and it may be given any name (excluding 
\code{definitions} and \code{code}) 
as long as it is the first start state in the file.  
Occasionally a lexer may need to match some patterns in one context
and other patterns in another context.  Multiple start states provide
a mechanism for achieving this.  The lexer always starts off in the
first start state encountered in the spec file.  Helper methods
are provided which lexer actions can call to switch the start state.

\section{Regular Expressions}
The patterns used in the PyLly lexer spec file use standard regular
expression conventions.  All patterns must be enclosed in quotes.
Quotes may appear within patterns as long as they are escaped to avoid
ending the pattern.
The following regular expression operators
are provided:

\begin{tableii}{l|l}{exception}{Expression}{Description}
  \lineii{( re )}  {Parentheses can be used to override precedence rules.}
  \lineii{re re}   {A regular expression can be concatenated with another.}
  \lineii{re | re} {Match either re.}
  \lineii{re +}    {Specify that re be matched one or more times.}
  \lineii{re *}    {Specify that re be matched zero or more times.}
  \lineii{re ?}    {Specify that re be matched zero or one times.}
  \lineii{.}       {Matches any character other than newline.}
\end{tableii}

Regular expression primitives are either characters, escaped characters
or character classes.  Most characters can be given directly.  Special
characters such as quote, question mark and vertical bar must be escaped
with a backslash (eg. \code{\e ?} for question mark) to prevent them from being
interpretted as regular expression operators.  The newline, tab, carriage
return and NUL characters are specified with \code{\e n}, \code{\e t}, 
\code{\e r} and \code{\e 0} respectively.

\section{Character Classes}
Character classes are enclosed in brackets and specify a set of characters
to match.  Within the brackets, single characters or ranges specified
as a low character, a dash and a high character, can be specified.  For
example \code{[a-z0]} specifies all characters from \code{"a"} to 
\code{"z"} and the character
\code{"0"}.  Escape characters can be used within character classes.  The
inversion of a character class is specified by using the 
\code{"\^{}"} character
immediately after the open brace: \code{[\^{}a-z]} specifies all characters
except those from \code{"a"} to \code{"z"}.  
In addition the following names can
be used within the character class (ie: \code{"[[:alpha:]0]"} for all the
alphabet and the \code{"0"} character):

\begin{tableii}{l|l}{exception}{Class Name}{Description}
  \lineii{[:alnum:]}   {Alphabet and number characters.}
  \lineii{[:alpha:]}   {Alphabet characters}
  \lineii{[:blank:]}   {Tab and space.}
  \lineii{[:cntrl:]}   {Control characters.}
  \lineii{[:digit:]}   {Number characters.}
  \lineii{[:graph:]}   {Characters that show up on the screen.}
  \lineii{[:lower:]}   {Lowercase alphabet characters.}
  \lineii{[:print:]}   {Characters that show up on the screen and space.}
  \lineii{[:punct:]}   {Punctuation characters.}
  \lineii{[:space:]}   {Any form of formatting character (spaces, newlines, ...).}
  \lineii{[:upper:]}   {Uppercase alphabet characters.}
  \lineii{[:xdigit:]}  {Hexadecimal number characters.}
\end{tableii}


\section{Pattern operators}
In addition to the normal regular expression operators and primitives,
the following may be used within the patterns in a start state section
(but not within the patterns of a definition section):


\begin{tableii}{l|l}{exception}{Expression}{Description}
  \lineii{\^{} re}    {The re is only matched at the start of the line.}
  \lineii{<{}<EOF>{}>}  {Matches only the end of file.}
\end{tableii}


\section{Action Code}
When a pattern is matched during lexing, the action code for the pattern
is invoked.  The code is executed with one argument, \code{self}, which refers
to the lexer class (eg. \code{pyggy.lexer.lexer}).  When the code is called,
the \code{self.value} variable contains a string of the characters that
were matched by the pattern.  The action code may alter this variable
as it sees fit.
If the action code returns \code{None}, lexing is continued, otherwise the
returned token is returned from the lexer to its caller.

The action code may make use of several methods in the lexer class:

\begin{tableii}{l|l}{exception}{Method}{Description}
  \lineii{\code{PUSHSTATE(statenum)}}  
	{When lexing continues, start in the specified start state.}
  \lineii{\code{POPSTATE()}} 
	{When lexing continues, start in the previous start state.}
  % XXX these need to be able to wrap around to multiple lines, but how?
  \lineii{\code{PUSHBACK(str)}} 
	{take the characters in \code{str} and
       push them back on the input stream so that they may be
       matched against when lexing continues.}
  \lineii{\code{ENQUEUE(tok, val)}} 
	{push a token onto the lexer queue.
       The next time a token is retrieved, it will be retrieved from
       the lexer queue prior to matching patterns from the input
       stream.  The lexer's \code{value} variable will be set from
       \code{val} when \code{tok} is returned.}
\end{tableii}

\section{Grammar}
% This was made by running pyggy -d1 pylly.pyg and
% then manually post-processing the output in vi.  If we
% do this often we might want to write a sed script.
\begin{productionlist}[pylly]
  \production{spec} {sect spec}
  \production{spec} {sect}
  \production{sect} {\token{definitions} \token{INDENT} deflist \token{DEDENT}}
  \production{sect} {\token{SRCCODE}}
  \production{sect} {statelist \token{INDENT} rulelist \token{DEDENT}}
  \production{statelist} {\token{IDENT}}
  \production{statelist} {statelist \token{,} \token{IDENT}}
  \production{deflist} {\token{IDENT} \token{\"} regexp \token{\"} \token{\e n}}
  \production{deflist} {deflist \token{IDENT} \token{\"} regexp \token{\"} \token{EOL}}
  \production{rulelist} {\token{\"} rulepat \token{\"} \token{SRCCODE}}
  \production{rulelist} {rulelist \token{\"} rulepat \token{\"} \token{SRCCODE}}
  \production{rulepat} {optanchor regexp}
  \production{rulepat} {\token{<{}<EOF>{}>}}
  \production{optanchor} {\token{\^{}}}
  \production{optanchor} {}
  \production{regexp} {reclause}
  \production{regexp} {regexp reclause}
  \production{reclause} {reclause \token{+}}
  \production{reclause} {reclause \token{*}}
  \production{reclause} {reclause \token{?}}
  \production{reclause} {reclause \token{|} reclause}
  \production{reclause} {\token{(} regexp \token{)}}
  \production{reclause} {\token{IDENT}}
  \production{reclause} {cclass}
  \production{reclause} {\token{CHAR}}
  \production{reclause} {\token{.}}
  \production{cclass} {\token{[} optinvert ranges \token{]}}
  \production{optinvert} {\token{\^{}}}
  \production{optinvert} {}
  \production{ranges} {range}
  \production{ranges} {ranges range}
  \production{range} {\token{CHAR}}
  \production{range} {\token{CHAR} \token{-} \token{CHAR}}
  \production{range} {\token{[:alnum:]}}
  \production{range} {\token{[:alpha:]}}
  \production{range} {\token{[:blank:]}}
  \production{range} {\token{[:cntrl:]}}
  \production{range} {\token{[:digit:]}}
  \production{range} {\token{[:graph:]}}
  \production{range} {\token{[:lower:]}}
  \production{range} {\token{[:print:]}}
  \production{range} {\token{[:punct:]}}
  \production{range} {\token{[:space:]}}
  \production{range} {\token{[:upper:]}}
  \production{range} {\token{[:xdigit:]}}
\end{productionlist}


	% spec file
    % lexer api/calling convention
\chapter{The PyGgy spec file}

The PyGgy spec file is used to specify grammars for a parsing engine.
PyGgy files are made up of a number of rules.  There are two types
of rules in the spec file: precedence rules and production rules.

\section{Production rules}
Production rules are used to specify productions in the grammar.
Each rule can specify several productions for a common non-terminal.
The rule is written with an identifier for the non-terminal (the
left hand side), the symbol \code{->} and a list of right hand sides seperated
by the \code{|} character, and terminated with the \code{;} character.  
Each right 
hand side is specified by zero or more terminals or non-terminals.
Terminals and non-terminals are represented by names starting with
a letter or an underscore and containing letters, underscores and
digits.  

Each right hand side may be given an optional precedence, which will
be described shortly.  Each right hand side may also be associated
with a block of action code.  Action code is specified with a colon
character followed by either a line of code, or a newline and an
indented block of code.

An example of a production is:
\begin{verbatim}
E -> TOK_ID :
        return symtab[kids[0]]
    | E TOK_ADD E :
        return kids[0] + kids[2]
    ;
\end{verbatim}


There may be many production rules in the
spec file.  The left hand side of the first production is taken
to be the start symbol of the grammar.

\section{Precedence rules}
Precedence rules are used to disambiguate ambiguous parses in the
grammar.  They can be used to disambiguate an otherwise ambiguous
grammar.  There are two precedence mechanisms.  The first is to assign
precedences and associativities to terminals.  This is done with
a precedence rule which specifies an associativity
followed by a list of non-terminals followed by a semicolon.
The associativity may be one of \code{\%left}, \code{\%right}
or \code{\%nonassoc}.  An example of a precedence rule is:

\begin{verbatim}
%left TOK_MUL TOK_ADD ;
\end{verbatim}

All tokens specified in this way are given priorities from highest
(most prefered) to lowest (lest prefered) in the order they are 
encountered in the
spec file.  The tokens are also assigned the designated associativity.
Any production making use of these terminals inherits their
associativity and precedence.

It is also possible to directly assign a precedence and an associativity
to a production.  This is done by prefixing a right hand side of a
production with a \code{\%prec} operator.  
For example, the following grammar has a
well known ambiguity:

\begin{verbatim}
Expr -> e
    ;
Stmt -> IF Expr THEN Stmt ELSE Stmt
    | IF Expr THEN Stmt 
    | s
    ;
\end{verbatim}

For example \code{IF e THEN s ELSE IF e THEN s ELSE s} can be parsed
either as \code{IF e THEN s ELSE [IF e THEN s ELSE s]\}} or
\code{IF e THEN s ELSE [IF e THEN s] ELSE s}.  We can specify that
the first \code{Stmt} production has higher precedence than the second
by specifying two dummy precedences and referencing them from
the right hand sides:

\begin{verbatim}
%nonassoc LONG SHORT ;

Stmt -> %prec(LONG) IF Expr THEN Stmt ELSE Stmt
    | %prec(SHORT) IF Expr THEN Stmt 
    | s
    ;

Expr -> e
    ;
\end{verbatim}

The resulting parser has no ambiguities and prefers
matching the longer statement whenever possible (binding the
else statement to the nearest if statement).  
See \file{exampmles/test3.py} for a complete example.

\section{Action Code}
Each production can have action code associated with it.  Action code
is specified with a colon followed by a line of code or an indented
block of code.

Actions associated with productions are not performed during parsing.
Rather these functions can be called during a post-processing pass of
the parse tree with the \code{proctree} function.  

The action code
is called with one argument named \code{kids} which is a list of
right hand side elements.  The value returned by the action code is
associated with the production it belongs to.  When the production
is used in the right hand side of another production in the parse
tree, its value will be passed in as an elements of the \code{kids}
argument for that production's action code.
See the API reference for more details.

If action code is not specified for a production, the default
action is used.  This action is simply \code{return kids}, which
returns a list of the right hand side elements of the derivation.
Some productions in the grammar are generated internally and have
their own actions associated with them, as will be seen in the next
section.

\section{EBNF}
PyGgy supports some extended BNF constructs to simplify the construction
of a grammar.  These constructs are:

\begin{tableii}{l|l}{exception}{EBNF Construct}{Description}
  \lineii{( rhs )} {Right hand side elements can be grouped.}
  \lineii{rhs *}   {Denotes zero or more repetitions of rhs.}
  \lineii{rhs +}   {Denotes one or more repetitions of rhs.}
  \lineii{rhs ?}   {Denotes that rhs is optional.}
\end{tableii}

Use of these constructs cause internal productions to be added
to the grammar.  These can be viewed by setting the debug flag to
\code{1} when parsing the spec file.
For example, the spec file:

\begin{verbatim}
lines -> line* ;
line -> A? | B+ | C D* ;
\end{verbatim}

results in the following grammar:

\begin{verbatim}
lines.clos1 ->
lines.clos1 -> lines.clos1 line
lines -> lines.clos1
line.opt2 ->
line.opt2 -> A
line -> line.opt2
line.posclos3 -> B
line.posclos3 -> line.posclos3 B
line -> line.posclos3
line.clos4 ->
line.clos4 -> line.clos4 D
line -> C line.clos4
\end{verbatim}


The internal productions contain action code that implement the 
obvious actions.  Productions generated with \code{"+"} or \code{"*"}
return a list of elements.  Productions generated with \code{"?"}
return the optional value or the value \code{None}.  Groupings of
elements return a list of the elements.



\section{The effect of precedence on Parser Construction}
During the construction of a parser, precedence rules are used
to disallow certain derivations from being considered.
The precedence rules setup a relationship between productions
in the grammar.  Two productions in the grammar will either have
no relation to each other, or one will be considered
{\em gt}, {\em right}, {\em left} or {\em nonassoc} in relation
to the other.  

If a production is {\em gt} than another, it
can never be used to derive the second production.  For
example if \code{[E -> E * E]} {\em gt} \code{[E -> E + E]} then
the derivations \code{[E -> [E -> E + E] * E]} and 
\code{[E -> E * [E -> E + E]]}
will never be allowed.

If a production is {\em right} with respect to another, then
the first production cannot derive the second one at the right-
most position of the first production.  For example if
\code{[E -> E * E]} {\em right} \code{[E -> E + E]} then the
derivation \code{[E -> [E -> E + E] * E]} is not allowed, but the
derivation \code{[E -> E * [E -> E + E]]} is.  Similarly the {\em left}
relation indicates that the first production cannot derive the
second at the left-most position.  The relationship {\em nonassoc}
is a combination of both the {\em right} and {\em left} relationships.

The relationship between productions is inferred from the
precedence rules and the precedence operators in a grammar
specification file.  The precedence rules setup a precedence
ordering among terminals.  Each precedence involving a terminal
that has a precedence associated with it inherits the precedence
of that symbol.  One production is {\em gt} another if it has
a higher precedence.  Productions relate to themselves with
the {\em right}, {\em left} or {\em nonassoc} relations if they
contain a terminal with the given associativity.

The restrictions placed on the derivation by the precedence rules
are used during parser construction in two ways.  First, it
is used to prevent certain items from being added to an itemset
if the item represents a precedence conflict.  Second, the
computation of the {\em FOLLOWS} set is modified to compute
the symbols that can follow each production, rather than the
symbols that can follow each symbol.  Symbols that indicate
a conflict are not added to the {\em FOLLOWS} set of a production.

\begin{seealso}
\seetitle[http://www.cs.uu.nl/groups/ST/Visser/DisambiguationFiltersForScannerlessGeneralizedLRParsers]{Disambiguation Filters for Scannerless Generalized LR Parsers}{The precedence system used by PyGgy is based on this paper.}
\end{seealso}



\section{Grammar}
% This was made by egrep '^[a-zA-Z]|[;|]' pyggy.pyg and post
% processing the output.  Sed script would be useful.
\begin{productionlist}[pyggy]
  \production{gram} {line | line gram}
  \production{line} {precoper idlist \token{;}}
  \production{line} {\token{\%rel} \token{ID} precoper \token{ID} \token{;}}
  \production{line} {\token{ID} \token{->} rhslist \token{;}}
  \production{line} {\token{code} \token{SRCCODE}}
  \production{precoper} {\token{\%left} | \token{\%right} | \token{\%nonassoc} | \token{\%pref} | \token{\%gt}}
  \production{rhslist} {optprec rhsellist optcode}
  \production{rhslist} {rhslist \token{ALT} optprec rhsellist optcode}
  \production{optcode} {\token{SRCCODE}}
  \production{optcode} {}
  \production{rhsellist} {rhsellist rhsel}
  \production{rhsellist} {}
  \production{rhsel} {\token{ID}}
  \production{rhsel} {\token{(} rhslist \token{)}}
  \production{rhsel} {rhsel \token{*}}
  \production{rhsel} {rhsel \token{+}}
  \production{rhsel} {rhsel \token{?}}
  \production{idlist} {idlist \token{ID}}
  \production{idlist} {}
  \production{optprec} {\token{\%prec} \token{(} \token{ID} \token{)}}
  \production{optprec} {}
\end{productionlist}


	% spec file
    % parser api/calling convention
	% references to papers

\chapter{Future Directions}

There's a lot left to be done.  This package is still young and
needs to be put through its paces.  There are still a number of
opportunities for improvement in the current implementation.
The grammars for the parser and lexer spec files are old and
do not yet make use of some of the newer features in PyGgy such
as the EBNF constructs.  The semantic actions of the parser are
performed in a post pass.  It might be possible to execute
these actions earlier during parsing and reduce the memory
requirements.  The internal representations used for the NFA,
DFA and shift-reduce tables may not be particularly efficient.
It is hard to know what works well and what does not without
trying it out.  Some profiling and tuning of PyGgy and PyLly
would be useful.

Beyond the obvious, there are a few things on my wish list for
making a better parsing system.  The first is to implement
scannerless parsing.  Scannerless parsing has a number of
advantages over parsing with a scanner (lexer).  PyGgy already
has many of the mechanisms that would be needed.  The second
is a mechanism for disambiguating parses during parsing.   Something
as simple as a disambiguating function could go a long way here.


\begin{seealso}
The following references were influential in the design and implementation
of this system.

\seetitle[http://www.cs.vu.nl/\%7edick/PTAPG.html]{Parsing Techniques - A Practical Guide}{A general overview of parsing technologies.}

\seetitle[http://www.cs.uu.nl/groups/ST/Visser/DisambiguationFiltersForScannerlessGeneralizedLRParsers]{Disambiguation Filters for Scannerless Generalized LR Parsers}{The precedence system used by PyGgy is based on this paper.}

\seetitle[http://citeseer.ist.psu.edu/rekers92parser.html]{Parser Generation for Interactive Environments}{The GLR parser in PyGgy is based on the parsing algorithm developed in section 1.5.1 of this paper.}

\seetitle[http://www.cs.uu.nl/groups/ST/Visser/ScannerlessGeneralizedLRParsing]{Scannerless Generalized-LR Parsing}{Building parsers without using lexers}

\seetitle[http://research.microsoft.com/research/pubs/view.aspx?msr\_tr\_id=MSR-TR-2003-32]{A Research C\# Compiler}{This paper describes a GLR-based compiler written by Microsoft Research with some novel features.  Their use of functions to disambiguate a parse is interesting.}

\seetitle[http://citeseer.ist.psu.edu/irwin01generated.html]{A Generated Parser of C++}{This paper describes a C++ parser built with a GLR parser.}

\seeurl{http://systems.cs.uchicago.edu/ply/}{PLY is another parser generator for Python.  Some examples in this text are derived from their examples.}

\seeurl{http://dparser.sourceforge.net/}{D-Parser is a GLR parser for C.  The ANSI-C example included with PyGgy is derived from a similar example for the D-Parser.}



\end{seealso}


	% wishlist
    % actions during parsing
    % disambiguation actions
    % scannerless parsing

\end{document}
