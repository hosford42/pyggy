\chapter{The PyGgy spec file}

The PyGgy spec file is used to specify grammars for a parsing engine.
PyGgy files are made up of a number of rules.  There are two types
of rules in the spec file: precedence rules and production rules.

\section{Production rules}
Production rules are used to specify productions in the grammar.
Each rule can specify several productions for a common non-terminal.
The rule is written with an identifier for the non-terminal (the
left hand side), the symbol \code{->} and a list of right hand sides seperated
by the \code{|} character, and terminated with the \code{;} character.  
Each right 
hand side is specified by zero or more terminals or non-terminals.
Terminals and non-terminals are represented by names starting with
a letter or an underscore and containing letters, underscores and
digits.  

Each right hand side may be given an optional precedence, which will
be described shortly.  Each right hand side may also be associated
with a block of action code.  Action code is specified with a colon
character followed by either a line of code, or a newline and an
indented block of code.

An example of a production is:
\begin{verbatim}
E -> TOK_ID :
        return symtab[kids[0]]
    | E TOK_ADD E :
        return kids[0] + kids[2]
    ;
\end{verbatim}


There may be many production rules in the
spec file.  The left hand side of the first production is taken
to be the start symbol of the grammar.

\section{Precedence rules}
Precedence rules are used to disambiguate ambiguous parses in the
grammar.  They can be used to disambiguate an otherwise ambiguous
grammar.  There are two precedence mechanisms.  The first is to assign
precedences and associativities to terminals.  This is done with
a precedence rule which specifies an associativity
followed by a list of non-terminals followed by a semicolon.
The associativity may be one of \code{\%left}, \code{\%right}
or \code{\%nonassoc}.  An example of a precedence rule is:

\begin{verbatim}
%left TOK_MUL TOK_ADD ;
\end{verbatim}

All tokens specified in this way are given priorities from highest
(most prefered) to lowest (lest prefered) in the order they are 
encountered in the
spec file.  The tokens are also assigned the designated associativity.
Any production making use of these terminals inherits their
associativity and precedence.

It is also possible to directly assign a precedence and an associativity
to a production.  This is done by prefixing a right hand side of a
production with a \code{\%prec} operator.  
For example, the following grammar has a
well known ambiguity:

\begin{verbatim}
Expr -> e
    ;
Stmt -> IF Expr THEN Stmt ELSE Stmt
    | IF Expr THEN Stmt 
    | s
    ;
\end{verbatim}

For example \code{IF e THEN s ELSE IF e THEN s ELSE s} can be parsed
either as \code{IF e THEN s ELSE [IF e THEN s ELSE s]\}} or
\code{IF e THEN s ELSE [IF e THEN s] ELSE s}.  We can specify that
the first \code{Stmt} production has higher precedence than the second
by specifying two dummy precedences and referencing them from
the right hand sides:

\begin{verbatim}
%nonassoc LONG SHORT ;

Stmt -> %prec(LONG) IF Expr THEN Stmt ELSE Stmt
    | %prec(SHORT) IF Expr THEN Stmt 
    | s
    ;

Expr -> e
    ;
\end{verbatim}

The resulting parser has no ambiguities and prefers
matching the longer statement whenever possible (binding the
else statement to the nearest if statement).  
See \file{exampmles/test3.py} for a complete example.

\section{Action Code}
Each production can have action code associated with it.  Action code
is specified with a colon followed by a line of code or an indented
block of code.

Actions associated with productions are not performed during parsing.
Rather these functions can be called during a post-processing pass of
the parse tree with the \code{proctree} function.  

The action code
is called with one argument named \code{kids} which is a list of
right hand side elements.  The value returned by the action code is
associated with the production it belongs to.  When the production
is used in the right hand side of another production in the parse
tree, its value will be passed in as an elements of the \code{kids}
argument for that production's action code.
See the API reference for more details.

If action code is not specified for a production, the default
action is used.  This action is simply \code{return kids}, which
returns a list of the right hand side elements of the derivation.
Some productions in the grammar are generated internally and have
their own actions associated with them, as will be seen in the next
section.

\section{EBNF}
PyGgy supports some extended BNF constructs to simplify the construction
of a grammar.  These constructs are:

\begin{tableii}{l|l}{exception}{EBNF Construct}{Description}
  \lineii{( rhs )} {Right hand side elements can be grouped.}
  \lineii{rhs *}   {Denotes zero or more repetitions of rhs.}
  \lineii{rhs +}   {Denotes one or more repetitions of rhs.}
  \lineii{rhs ?}   {Denotes that rhs is optional.}
\end{tableii}

Use of these constructs cause internal productions to be added
to the grammar.  These can be viewed by setting the debug flag to
\code{1} when parsing the spec file.
For example, the spec file:

\begin{verbatim}
lines -> line* ;
line -> A? | B+ | C D* ;
\end{verbatim}

results in the following grammar:

\begin{verbatim}
lines.clos1 ->
lines.clos1 -> lines.clos1 line
lines -> lines.clos1
line.opt2 ->
line.opt2 -> A
line -> line.opt2
line.posclos3 -> B
line.posclos3 -> line.posclos3 B
line -> line.posclos3
line.clos4 ->
line.clos4 -> line.clos4 D
line -> C line.clos4
\end{verbatim}


The internal productions contain action code that implement the 
obvious actions.  Productions generated with \code{"+"} or \code{"*"}
return a list of elements.  Productions generated with \code{"?"}
return the optional value or the value \code{None}.  Groupings of
elements return a list of the elements.



\section{The effect of precedence on Parser Construction}
During the construction of a parser, precedence rules are used
to disallow certain derivations from being considered.
The precedence rules setup a relationship between productions
in the grammar.  Two productions in the grammar will either have
no relation to each other, or one will be considered
{\em gt}, {\em right}, {\em left} or {\em nonassoc} in relation
to the other.  

If a production is {\em gt} than another, it
can never be used to derive the second production.  For
example if \code{[E -> E * E]} {\em gt} \code{[E -> E + E]} then
the derivations \code{[E -> [E -> E + E] * E]} and 
\code{[E -> E * [E -> E + E]]}
will never be allowed.

If a production is {\em right} with respect to another, then
the first production cannot derive the second one at the right-
most position of the first production.  For example if
\code{[E -> E * E]} {\em right} \code{[E -> E + E]} then the
derivation \code{[E -> [E -> E + E] * E]} is not allowed, but the
derivation \code{[E -> E * [E -> E + E]]} is.  Similarly the {\em left}
relation indicates that the first production cannot derive the
second at the left-most position.  The relationship {\em nonassoc}
is a combination of both the {\em right} and {\em left} relationships.

The relationship between productions is inferred from the
precedence rules and the precedence operators in a grammar
specification file.  The precedence rules setup a precedence
ordering among terminals.  Each precedence involving a terminal
that has a precedence associated with it inherits the precedence
of that symbol.  One production is {\em gt} another if it has
a higher precedence.  Productions relate to themselves with
the {\em right}, {\em left} or {\em nonassoc} relations if they
contain a terminal with the given associativity.

The restrictions placed on the derivation by the precedence rules
are used during parser construction in two ways.  First, it
is used to prevent certain items from being added to an itemset
if the item represents a precedence conflict.  Second, the
computation of the {\em FOLLOWS} set is modified to compute
the symbols that can follow each production, rather than the
symbols that can follow each symbol.  Symbols that indicate
a conflict are not added to the {\em FOLLOWS} set of a production.

\begin{seealso}
\seetitle[http://www.cs.uu.nl/groups/ST/Visser/DisambiguationFiltersForScannerlessGeneralizedLRParsers]{Disambiguation Filters for Scannerless Generalized LR Parsers}{The precedence system used by PyGgy is based on this paper.}
\end{seealso}



\section{Grammar}
% This was made by egrep '^[a-zA-Z]|[;|]' pyggy.pyg and post
% processing the output.  Sed script would be useful.
\begin{productionlist}[pyggy]
  \production{gram} {line | line gram}
  \production{line} {precoper idlist \token{;}}
  \production{line} {\token{\%rel} \token{ID} precoper \token{ID} \token{;}}
  \production{line} {\token{ID} \token{->} rhslist \token{;}}
  \production{line} {\token{code} \token{SRCCODE}}
  \production{precoper} {\token{\%left} | \token{\%right} | \token{\%nonassoc} | \token{\%pref} | \token{\%gt}}
  \production{rhslist} {optprec rhsellist optcode}
  \production{rhslist} {rhslist \token{ALT} optprec rhsellist optcode}
  \production{optcode} {\token{SRCCODE}}
  \production{optcode} {}
  \production{rhsellist} {rhsellist rhsel}
  \production{rhsellist} {}
  \production{rhsel} {\token{ID}}
  \production{rhsel} {\token{(} rhslist \token{)}}
  \production{rhsel} {rhsel \token{*}}
  \production{rhsel} {rhsel \token{+}}
  \production{rhsel} {rhsel \token{?}}
  \production{idlist} {idlist \token{ID}}
  \production{idlist} {}
  \production{optprec} {\token{\%prec} \token{(} \token{ID} \token{)}}
  \production{optprec} {}
\end{productionlist}

